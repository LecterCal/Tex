\chapter{收益率}
\begin{introduction}
	\item 收益率
	\item 基金的利息度量
	\item 再投资
	\item 基金
\end{introduction}
\section{收益率}
已知一个项目的现金流,如何评价其收益水平的高低?\\
\subsection{净现值}
(1) 净现值法 (net present value, NPV)
\[
N P V(i)=\sum_{t=0}^{n} v^{t} R_{t}=\text { 资金流入现值 }-\text { 资金流出现值 }
\]
\subsection{收益率法}
资金流入的现值与资金流出的现值之差就是净现值,所以收益率也是使得净现值等手零的利率:
\[
N P V(i)=\sum_{t=0}^{n} v^{t} R_{t}=0
\]
\section{基金的利息度量}
\begin{definition}{币值加权收益率}
\noindent	Suppose the following information is known:
	(i) the balance in a fund at the start of the year is $A$ \\
	(ii) for $0<t_{1}<t_{2}<\cdots<t_{n}<1,$ the net deposit at time $t_{k}$ is amount $C_{k}$ (positive for a net deposit, negative for a net withdrawal), and \\
	(iii) the balance in the fund at the end of the year is $B$
	Then the net amount of interest earned by the fund during the year is $I=B-\left[A+\sum_{k=1}^{n} C_{k}\right],$ and the dollar-weighted rate of return earned by the fund for the year is
	\[
	\frac{I}{A+\sum_{k=1}^{n} C_{k}\left(1-t_{k}\right)}
	\]
	\\---度量投资者的业绩
\end{definition}
\begin{remark}
	$(Ia)_{\angles{n}}=\frac{\ddot{a}_{\bar{n}}-n v^{n}}{i}$
\end{remark}
\begin{definition}{时间加权收益率}
\noinent Suppose the following information is known:\\
	(i) the balance in a fund at the start of the year is $A$\\
	(ii) for $0<t_{1}<t_{2}<\cdots<t_{n}<1,$ the net deposit at time $t_{k}$ is amount $C_{k}($ positive for a net deposit, negative for a net withdrawal)\\
	(iii) the value of the fund just before the net deposit at time $t_{k}$ is $F_{k},$ and\\
	(iv) the balance in the fund at the end of the year is $B$
	The time-weighted return rate earned by the fund for the year is
	\[
	\left[\frac{F_{1}}{A} \times \frac{F_{2}}{F_{1}+C_{1}} \times \frac{F_{3}}{F_{2}+C_{2}} \times \cdots \times \frac{F_{k}}{F_{k-1}+C_{k-1}} \times \frac{B}{F_{k}+C_{k}}\right]-1
	\]	
	\\度量基金经理人的业绩
\end{definition}	
\section{再投资}
\begin{example}
期初投资 1 元,投资期限为 $n$ 年, 年实际利率为 $i$ ,如果每年产生的利息按照年实际利率 $j$ 进行用投资, 试计算在第 $n$ 年末的累积值与该项投资的年平均收益率。
\end{example}
\begin{solution}
(1) 该投资在第 $n$ 年末的黑积值为 $1+i \cdot s_{\angles{n} j}$\\
(2) 假设该项投资的年平均收益率为 X,则由价值方程
\[
1+i \cdot s_{\angles{n} j}=(1+x)^{n} \Longrightarrow x=\left(1+i \cdot s_{\angles{n} j}\right)^{\frac{1}{n}}-1
\]
特别地,当 $j=i$ 时 $, x=i$
\end{solution}
\subsection{修正收益率}
如果再投资的利率与筹集资金的利率不同时,则评价项且应该使用修正收益率 (modified rate of internal rate) 在计算修正收益率时\\
对资金流出使用筹资的利率计算其现值\\
对资金流入使用再投资的利率计算其累积值
\section{基金}
\noindent 基金的收益:
\\收入:利息(债券,银行存款)、股息、资本利得(股票价格上涉)
\\净收益: 收入 - 费用
基金包括不同时期投资,如何把基金收入分配给不同时期的投资?
\\分配方法:
\\投资组合法 (portfolio method)
\\投资年度法 (investment year method)