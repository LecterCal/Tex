\chapter{债务价值分析}
\begin{introduction}
	\item 定价公式
	\item 等额偿债基金
	\item 变额分期偿还
\end{introduction}
\section{符号一览}
\noindent $P:$ 债券价格 (bond price) \\
$i$ :债券的到期收益率(yield-to-maturity rate),即投资人购买债券者购买债券所要求的收益率\\
$F :$ 债券的面值 (par value,face amount, nominal value),即债券到时支付给债券持有人的金额,也称为票面价值或到期值\\
$r:$ 债券的息票率 (coupon rate per payment period)\\
$rF$ 息票收入 \\
$i_{c}:$ 债券的当期收益率: $i_{c}=\frac{r F}{P}$\\
$C :$ 债券的偿还值 (redemption payment),通常等于侦券面值F\\
$n$ : 息票的支付次数 (number of coupon payments)\\
$g:$ 债券的修正息票率, $g=\frac{r F}{C}$\\
债券定价原理:到期偿还值+债券未来息票
$$P=rFa_{\angles{n}}+Cv^n$$
\begin{note}
\begin{aligned}
\frac{\partial P}{\partial i} &=-\left(\sum_{t=1}^{n} r F t v^{t+1}+C n v^{n+1}\right)<0 \\
\frac{\partial^{2} P}{\partial i^{2}} &=\sum_{t=1}^{n} r F t(t+1) v^{t+2}+C n(n+1) v^{n+2}>0
\end{aligned}
\end{note}
\noindent 定价公式主要包含:\\
(1) 基本公式 \\
(2) 溢价公式 \\
(3) 基价公式 \\
(4) Makeham 公式
$$P=\left\{\begin{array}{ll}
r F a_{\angles{n}}+C v^{n} & \text { 基本公式 } \\
C+C(g-i) a_{\angles{n}} & \text { 溢价公式 } \\
G+(C-G) v^{n} & \text { 基价公式 } \\
\frac{g}{i}(C-K)+K & \text { Makeham 公式 }
\end{array}\right.$$
(1) 息票率:
$$r=\frac{\text { 息票收入 }}{\text { 面值 }(F)}$$
(2) 修正息票率:
$$g=\frac{\text { 息票收入 }}{\text { 偿还值 }(C)}$$
(3) 到期收益率:
$$i=\frac{\text { 息票收入 }}{\text { 基价 }(G)}$$
(4) 当期收益率:
$$i_{c}=\frac{\text { 息票收入 }}{\text { 债券价格 }(P)}$$