\chapter{利率风险}
\begin{definition}{利息力}
\noindent		设可积函数连续可导,则称
\[
\delta_{t}=\frac{a^{\prime}(t)}{a(t)}=[\ln a(t)]^{\prime}
\]
为时刻 t 的利息力
		\end{definition}
		\\ 衡量利息增长速率与利息本身大小的比值
		很有意思的是,我们把t当做横坐标的时候,相当于t是定下来的,以这个为标准来用利息力来计算累积函数
		$$\int_{0}^{t} \delta_{s} d s=\int_{0}^{t} \frac{a^{\prime}(s)}{a(s)} d s=\int_{0}^{t}[\ln a(s)]^{\prime} d s=\ln a(t)$$
\section{Macaulay duration}
	\begin{definition}{Macaulay duration}
$$D_{\text {麦 }}=\frac{\sum_{t>0} t \cdot R_{t} e^{-\delta t}}{\sum_{t>0}  \cdot R_{t} e^{-\delta t}}$$
\end{definition}
\begin{note}
一笔 $n$ 年期贷款,年实际利率为 $i,$ 按年等额分期偿还,求该笔贷款的麦考利久期。
\end{note}
\begin{solution}
$$\begin{aligned}
D_{\text {麦 }} &=\frac{\sum\limits_{t>0} t \cdot R(1+i)^{-t}}{\sum\limits_{t>0} R(1+i)^{-t}}=\frac{(I a)_{\angles{n }}}{a_{\angles{n }}}} \\
&=\frac{\ddot{a}_{\angles{n}}-n v^{n}}{i \cdot a_{\angles{n }}}=\frac{(1+i) a_{\angles{n }}}{i \cdot a_{\angles{n }}}-\frac{n v^{n}}{i \cdot \frac{1-v^{n}}{i}} \\
&=\frac{1+i}{i}-\frac{n}{(1+i)^{n}-1}
\end{aligned}$$
\end{solution}
\subsection{修正久期}
\begin{definition}{修正久期}
$$D=\frac{D_{\text {麦 }}}{1+\frac{y}{m}}$$
\end{definition}
\begin{note}
	债券面值为F,期限为n,到期时按面值偿还,年息票率为r,到期收益率r,求$D_{\text {麦 }$\\
		$$D_{\text {麦 }}=\frac{\sum_{t=1}^{n}tv^trF+nv^nF}{\sum_{t=1}^{n}rFv^t+v^nF}$$
		$$=\frac{\sum_{t=1}^{n}tv^tr+nv^n}{\sum_{t=1}^{n}rv^t+v^n}$$
\end{note}
\begin{exercise}
某 2 年期债券的面值为 1000 元,年息票率为 8%,每半年末支付一次利息,债券到期按面值偿还。该债券每半年复利一次的到期收益率为4%,请计算该债券的修正久期
\end{exercice}
\begin{solution}
该债券的价格为
$$P=2 \times 40 a_{\angles{2}4 \%}^{(2)}+1000 \times(1+4 \% / 2)^{-4}=1076.16(\text { 元 })
$$
麦考利久期为
$$D_{\text {麦 }}=\frac{\sum_{t>0} t \cdot R_{t}(1+i)^{-t}}{P}=\frac{2036.19}{1076.16}=1.89$$
修正久期:$$D=\frac{D_{\text {麦 }}}{1+i^{(2)} / 2}=\frac{1.89}{1+4 \% / 2}=1.86$$
\end{solution}