\documentclass[cn,11pt,chinese,black]{elegantbook}
\title{用迭代法求解非线性方程的matlab实现}
\author{组长:李宗翰 ,组员:蒋佩禧,袁浩然}
\cover{cover4.png}
% 本文档命令
\usepackage{array}
\newcommand{\ccr}[1]{\makecell{{\color{#1}\rule{1cm}{1cm}}}}
\usepackage{mathpazo} 
\usepackage{algorithm}  
\usepackage{algpseudocode}  
\usepackage{amsmath}  
\renewcommand{\algorithmicrequire}{\textbf{Input:}}  % Use Input in the format of Algorithm  
\renewcommand{\algorithmicensure}{\textbf{Output:}} % Use Output in the format of Algorithm  
\begin{document}  %开始写文章
	\maketitle
	\chapter{用迭代法求解非线性方程的matlab实现}
	\section{算法结构}
	对于非线性方程 $f(x)=0,$ 若 $f^{\prime}\left(x_{k}\right) \neq 0,$ 建立迭代格式
	\[
	x_{k+1}=x_{k}-\frac{f\left(x_{k}\right)}{f^{\prime}\left(x_{k}\right)}, k=0,1, \cdots
	\]
	相应的迭代函数
	\[
	\varphi(x)=x-\frac{f(x)}{f^{\prime}(x)}
	\]
	\subsection{计算步骤}
\noindent	1. 给出 $x_{0}, \varepsilon, N$ \\
	2. 计算 $x_{1}=x_{0}-\frac{f\left(x_{0}\right)}{f^{\prime}\left(x_{0}\right)}$ \\
	3. 若 | $x_{1}-x_{0} \mid<\varepsilon$ 则转 $4, \quad$ 否则 $x_{0}=x_{1},$ 转步骤 2 \\
	4. 输出 $x_{1},$ 结束
\begin{lstlisting}[ language=Matlab] 
function Newton(f,x0,e)
syms x
y=f(x);  %把f(x)转化为符号函数y
df=matlabFunction(gradient(y)); %对y求导,并把导数转化为句柄函数df
x=x0-feval(f,x0)/feval(df,x0);
k=0;
while norm(x-x0)>e   %判断误差
 k=k+1;
 x0=x;
 x=x0-feval(f,x0)/feval(df,x0);
 if k==500        %限制循环次数
 error('迭代次数过多可能不收敛')
 else 
fprintf('%d%f\n',k,x) %输出结果
end
end
\end{lstlisting}
\subsection{进一步改进}
\begin{lstlisting}[ language=Matlab] 
function [c,E,fc]=newton1(f,a,b,error,n_iter)
if sum(size(a))~=2 || sum(size(b))~=2
error('错误'); 
end  
end
if (nargin<4) 
error=0.00001;
n_iter=1000;
elseif (nargin<5) 
n_iter=1000;
if sum(size(error))~=2
error('.'); 
end
else
if sum(size(error))~=2
error('.'); 
end
if round(n_iter)-n_iter~=0
error('.');
end
end
if subs(f,a)*subs(f,b)>0
error('');
end
%%  Newton
if subs(f,a)*subs(diff(f,2),a)>0
x=a; 
else
x=b; 
end
E=b-a;
xa=x; i=0;
RES(1,1)=x; ERR(1,1)=E;

while (E>=error) && (i<n_iter)
if subs(f,x)==0
fprintf(: %f',x);
break;
else
x=xa-( subs(f,xa) / subs( diff(f),xa ) );
end
E=abs(x-xa);
xa=x;
i=i+1;
RES(i+1,1)=x; ERR(i+1,1)=E;
end
disp('--------|----------|------------|');
fprintf('Iter.Aprox.Error.\n');
disp('--------|----------|------------|');
fprintf('%-10s%-11.4f%-12.4f\n','ini',RES(1,1),ERR(1,1));
for i=2:max(size(RES))
fprintf('%-10.0d%-11.4f%-12.4f\n',i-1,RES(i,1),ERR(i,1));
end
disp('--------|----------|------------|');
c=x; fc=subs(f,x);
\end{lstlisting}
$$f(x)=x^2-1$$
\begin{lstlisting}
 f=sym('x^2-1');
 [c,E,fc]=newton1(f,0,3,0.005,20);
\end{lstlisting}
$$\begin{array}{lll}
\text { Iter. } & \text { Aprox. } & \text { Error. } \\
\text {  } & 3.0000 & 3.0000 \\
1 & 1.6667 & 1.3333 \\
2 & 1.1333 & 0.5333 \\
3 & 1.0078 & 0.1255 \\
4 & 1.0000 & 0.0078 \\
5 & 1.0000 & 0.0000
\end{array}$$
\begin{lstlisting}
f=sym('x^5+x^4-1');
 [c,E,fc]=newton1(f,0,3,0.005,20);
\end{lstlisting}
$$\begin{array}{lll}
\text { Iter. } & \text { Aprox. } & \text { Error. } \\
\text {  } & 3.0000 & 3.0000 \\
1 & 2.3704 & 0.6296 \\
2 & 1.8711 & 0.4992 \\
3 & 1.4803 & 0.3908 \\
4 & 1.1853 & 0.2950 \\
5 & 0.9849 & 0.2005 \\
6 & 0.8831 & 0.1017 \\
7 & 0.8580 & 0.0251 \\
8 & 0.8567 & 0.0014
\end{array}$$
\section{牛顿法和二分法做对比}
$$x^{4}-50 x^{3}-6000=0$$
分别根据牛顿法和二分法进行实验
$$\begin{array}{|c|c|c|}
\hline \text { 迭代数 } & \text { 每次迭代初始值:x0 }  & \text { 每次迭代输出值x1 }: \ \\
\hline 1 & 50.0000000000000 & 52.6440010000000 \\
\hline 2 & 52.4057080000000 & 52.4087180000000 \\
\hline 3 & 52.4067310000000 & 52.4087310000000 \\
\hline
\end{array}$$
\end{document}
