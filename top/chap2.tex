\chapter{分离公理}
\begin{introduction}
	\item $T_1$公理
	\item $T_2$公理
	\item $T_3$公理
	\item $T_4$公理
\end{introduction}
\begin{definition}{$T_{1},T_{2}$公理}
\noindent $T_{1}$公理:$\quad$ 任何两个不同的点$x, y,$ 有 $x$ 的开邻域不包含 $y, y$ 有开邻域不包含$x$\\
$T_{2}$公理: $\quad$ 任何两个不同的点有不相交的开邻域.
\end{definition}

\chapter{可数公理}
\chapter{拓扑基}
\begin{definition}{拓扑基}
\noindent If $X$ is a set, a basis for a topology on $X$ is a collection $\mathscr{B}$ of subsets of $X$ (called basis elements) such that\\
(1) For each $x \in X$, there is at least one basis element $B$ containing $x$.\\
(2) If $x$ belongs to the intersection of two basis elements $B_{1}$ and $B_{2},$ then there is a basis element $B_{3}$ containing $x$ such that $B_{3} \subset B_{1} \cap B_{2}$ \\
If $\mathscr{B}$ satisfies these two conditions, then we define the topology $\boldsymbol{\mathcal { T }}$ generated by $\mathcal{B}$ as follows: $A$ subset $U$ of $X$ is said to be open in $X$ (that is, to be an element of $\mathcal{T}$ ) if for each $x \in U,$ there is a basis element $B \in \mathscr{B}$ such that $x \in B$ and $B \subset U .$ Note that each basis element is itself an element of $\mathcal{T}$
\end{definition}